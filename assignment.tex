\documentclass[a4paper,10pt,english,swedish]{article}

% Customized for UNIX and XeTeX (Does NOT work with PDFLaTeX)

% Adds support for code listings
\usepackage{listings}

% Adds advanced syntax highlighting support via "pygments"
\usepackage{minted}
% You need to install minted.sty and the "pygments" Python package for
% this!

% Adds page headers
\usepackage{fancyhdr}

% Adds direct UTF-8 support in source files for XeTeX
\usepackage{xunicode}

% You think Babel is good for language support? Hah, try this!
\usepackage{polyglossia}

% Adds PDF features
\usepackage[pdfusetitle,bookmarks=true,
 bookmarksnumbered=true,bookmarksopen=false,
 breaklinks=false,pdfborder={0 0 0},backref=false,
 colorlinks=false]{hyperref}

% Include author-specific information from the external file
\def\theauthor{}
\def\assignmentnr{}
\def\coursename{}
\def\courseleader{}
\def\course{}
\def\groupnr{}
\def\thedate{\today}


% Fontspecs
% Using XeTeX? Good! These awesome fonts are awesome
% (But they need to be installed on your system as .ttf or
%  .otf files):
\usepackage{fontspec}
\setmainfont[Mapping=tex-text,Numbers=OldStyle]{Calluna}
\setsansfont[Mapping=tex-text]{Calluna Sans}
\setmonofont{Droid Sans Mono}

% Alternate fonts - More "swirly" fonts:
%\usepackage{fontspec}
%\setmainfont[Mapping=tex-text,Numbers=OldStyle]{Fontin}
%\setsansfont[Mapping=tex-text]{Fontin Sans}
%\setmonofont{Droid Sans Mono}

% Alternate fonts - Linux:
%\usepackage{fontspec}
%\setmainfont[Mapping=tex-text,Numbers=OldStyle]{Droid Serif}
%\setsansfont[Mapping=tex-text]{Droid Sans}
%\setmonofont{Droid Sans Mono}

% Alternate fonts - Windows:
%\usepackage{fontspec}
%\setmainfont[Mapping=tex-text,Numbers=OldStyle]{Cambria}
%\setsansfont[Mapping=tex-text]{Calibri}
%\setmonofont{Consolas}

% Alternate fonts - Windows OldSkool:
%\usepackage{fontspec}
%\setmainfont[Mapping=tex-text,Numbers=OldStyle]{Times New Roman}
%\setsansfont[Mapping=tex-text]{Arial}
%\setmonofont{Courier New}

% Not using XeTeX? Oh well, still use some awesome fonts:
%\usepackage{bookman}
%\renewcommand{\sfdefault}{lmss}
%\renewcommand{\ttdefault}{lmtt}

% Listings
\renewcommand\listingscaption{Programkod}
\renewcommand\listoflistingscaption{Programkodsförteckning}

% Mint
% You can select a different style here:
%\usemintedstyle{monokai}
%\usemintedstyle{manni}
%\usemintedstyle{perldoc}
%\usemintedstyle{borland}
%\usemintedstyle{colorful}
%\usemintedstyle{default}
%\usemintedstyle{murphy}
%\usemintedstyle{vs}
%\usemintedstyle{trac}
%\usemintedstyle{tango}
%\usemintedstyle{fruity}
%\usemintedstyle{autumn}
%\usemintedstyle{bw}
%\usemintedstyle{emacs}
%\usemintedstyle{vim}
\usemintedstyle{friendly}
%\usemintedstyle{native}

% Fancy background color
\definecolor{codebg}{rgb}{0.95,0.95,0.95}

% Fancy line numbers
\renewcommand{\theFancyVerbLine}{
  \textcolor[rgb]{0.5,0.5,0.5}{\scriptsize
    \oldstylenums{\arabic{FancyVerbLine}}}}

% Add support for Java code sections.
\newminted{java}{bgcolor=codebg,linenos=true,numbersep=6pt,
  mathescape=true,texcl=true,gobble=2}
\newmintedfile{java}{bgcolor=codebg,linenos=true,numbersep=6pt,
  stepnumber=5,mathescape=true,texcl=true}
\newmint{java}{bgcolor=codebg,mathescape=true,texcl=true}

% Header and footer
\pagestyle{fancy}
\lhead{\course\ -\ Inlämning \assignmentnr}
\rhead{\theauthor\ -\ \thedate}
\renewcommand{\headrulewidth}{0pt}

% Language
\setdefaultlanguage{swedish}
% Surround english text with "\textenglish{}"
\setotherlanguage{english}

% Paragraphs
\setlength{\parskip}{\smallskipamount}
\setlength{\parindent}{0pt}

% Metadata
\title{Inlämning \assignmentnr\ - \course\ \coursename}
\date{\thedate}
\author{\theauthor}
\begin{document}

% Remove this to not have a cover sheet.
\thispagestyle{empty}

\textbf{\large\course\ \coursename}

\begin{description}
\item[Uppgift nummer:] \assignmentnr
\item[Namn:] \theauthor
\item[Grupp nummer:] \groupnr
\item[Övningsledare:] \courseleader
\end{description}

\rule[0.5ex]{1\columnwidth}{1pt}

\textbf{Betyg:} ..... \hfill \textbf{Datum:} .............. \hfill \textbf{Rättad
av:} ........................................
\newpage

% Include this to get a "normal article"
%\maketitle

\setcounter{page}{1}

\section{Vad är det här?}

Det här är alltså ett exempel på hur man kan lämna in sina INDA-labbar
på ett snyggt sätt (så att man får guldstjärnor av sin
asse). \LaTeX-mallen är förhoppningsvis kommenterad tillräckligt så
att man ser vad det mesta betyder så att du kan ändra den efter eget
behag.

Kom ihåg att byta ut ditt namn, kod och innehåll innan du lämnar in
inlämningen.

\section{Källkod}

På det här sättet kan man inkludera en hel fil, så att du inte behöver
hålla på och kopiera det in i \LaTeX. Notera att detta exempel
förutsätter att nämnda kod-fil ligger i samma mapp som denna
.tex-fil.

\subsection{Math.java}

\javafile{Math.java}

\subsection{Inkludering av kod direkt i \LaTeX}

Ibland ska bara mindre småsaker redovisas, då kan det vara smidigare
att lägga koden direkt i detta .tex-dokument som i detta exempel.

För att skriva ut ``Hello, World!'' i Java så kan man lämpligtvis köra
följande metodanrop.

% Istället för /.../ kan man använda vilken separator som helst, ex. :...:
\java/System.out.println("Hello, World!");/

Alternativt:

\begin{javacode}
  import static System.out;
  // ...
  out.println("Hello World!");
\end{javacode}

\end{document}

